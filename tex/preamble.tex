%Paquetes

\usepackage[dvipsnames]{xcolor}
\usepackage{anyfontsize}

\usepackage[spanish]{babel}
\usepackage{amsmath}
\usepackage{amsfonts}
\usepackage{amssymb}
\usepackage{amsthm}
%\usepackage{pxfonts}
\usepackage[newfloat=true]{minted}

\usepackage{etoolbox}
\usepackage{float}
\usepackage{multirow}
\usepackage{graphicx}
\usepackage{subcaption}
\usepackage{tikz}
\usepackage{hyperref}
\usepackage{calrsfs}
\usepackage{mathrsfs}
%\usepackage{bookmark}
\usepackage{caption}
\usepackage{mathtools}

%Fuente
\usepackage{charter}

\usetikzlibrary{shapes.geometric}
\usetikzlibrary{calc}
\usetikzlibrary{cd}

%%%%% DIR FOTOS %%%%
\graphicspath{ {./fotos/} }

%%%%% ENTORNOS %%%%%
\newtheorem{teorema}{Teorema}
\newtheorem{proposicion}{Proposici\'on}
\newtheorem{lema}{Lema}
\newtheorem{corolario}{Corolario}
\newtheorem*{idea}{Idea}
\newtheorem{problema}{Problema}

\theoremstyle{definition}
\newtheorem{definicion}{Definici\'on}[section]
\newtheorem{ejemplo}{Ejemplo}[section]
\newtheorem{ejercicio}{Ejercicio}[section]
\newtheorem{remark}{Remark}[section]
\newtheorem{obs}{Observaci\'on}[section]
\newtheorem*{sol}{Soluci\'on}

\AtEndEnvironment{sol}{\null\hfill\qed}

\AtEndEnvironment{definicion}{\null\hfill\textasteriskcentered}
\AtEndEnvironment{remark}{\null\hfill\textasteriskcentered}
\AtEndEnvironment{obs}{\null\hfill\textasteriskcentered}

\AtEndEnvironment{ejemplo}{\null\hfill\dag}
\AtEndEnvironment{ejercicio}{\null\hfill\dag}

%%%%% COMANDOS %%%%%

\SetupFloatingEnvironment{listing}{name=C\'odigo}

%Black-board Letters
\newcommand{\C}{\mathbb{C}}
\newcommand{\R}{\mathbb{R}}
\newcommand{\Q}{\mathbb{Q}}
\newcommand{\Z}{\mathbb{Z}}
\newcommand{\N}{\mathbb{N}}
\newcommand{\K}{\mathbb{K}}

%
\newcommand{\preg}{\textquestiondown}
\newcommand{\fl}[1]{\overrightarrow{#1}}
%

\newcommand{\DS}{\displaystyle}
%Ceil/floor
\DeclarePairedDelimiter\ceil{\lceil}{\rceil}
\DeclarePairedDelimiter\floor{\lfloor}{\rfloor}

%Convex hull/closed convex hull
\DeclareMathOperator{\ch}{co}
\newcommand{\cch}{\overline{\ch}}

%Linear span
\DeclareMathOperator{\lspan}{span}
\newcommand{\cspan}{\overline{\lspan}}

%Distance/diameter
\DeclareMathOperator{\dist}{dist}
\DeclareMathOperator{\diam}{diam}

%Diagonal operator
\DeclareMathOperator{\diag}{diag}

%Locally/support
\DeclareMathOperator{\loc}{loc}
\DeclareMathOperator{\supp}{supp}

%interior/domain/id
\DeclareMathOperator{\inter}{int}
\DeclareMathOperator{\dom}{dom}
\DeclareMathOperator{\id}{id}

%weak limits
\DeclareMathOperator{\wlim}{\omega-lim}
\DeclareMathOperator{\wslim}{\omega^*-lim}

%differentials and integrals
\newcommand{\dint}{\displaystyle\int}
\newcommand{\diint}{\displaystyle\iint}
\newcommand{\diiint}{\displaystyle\iiint}

\newcommand{\df}{{\rm d}f}
\newcommand{\dx}{{\rm d}x}
\newcommand{\dy}{{\rm d}y}
\newcommand{\dz}{{\rm d}z}
\newcommand{\dt}{{\rm d}t}

\newcommand{\dxy}{{\rm d}(x,y)}
\newcommand{\dxyz}{{\rm d}(x,y,z)}

\newcommand{\ds}{{\rm d}s}
\newcommand{\dS}{{\rm d}S}
\newcommand{\dV}{{\rm d}V}
\newcommand{\dA}{{\rm d}A}

\newcommand{\dpf}{\partial f}
\newcommand{\dpx}{\partial x}
\newcommand{\dpy}{\partial y}
\newcommand{\dpz}{\partial z}

%%%%% ITEMS %%%%
%\newcommand{\hitem}{\item[\color{MidnightBlue}]} 
%\newcommand{\eitem}{\item[\color{LimeGreen}]}
